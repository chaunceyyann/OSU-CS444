\documentclass[letterpaper,10pt]{article}

\usepackage{graphicx}                                        
\usepackage{amssymb}                                         
\usepackage{amsmath}                                         
\usepackage{amsthm}                                          

\usepackage{alltt}                                           
\usepackage{float}
\usepackage{color}
\usepackage{url}

\usepackage{balance}
\usepackage[TABBOTCAP, tight]{subfigure}
\usepackage{enumitem}
\usepackage{pstricks, pst-node}

\usepackage{geometry}
\geometry{textheight=8.5in, textwidth=6in}
\usepackage{booktabs}

%random comment

\newcommand{\cred}[1]{{\color{red}#1}}
\newcommand{\cblue}[1]{{\color{blue}#1}}
\newcommand{\tab}{\hspace*{2em}} % for tabbing
\newcommand{\toc}{\tableofcontents}

\usepackage{hyperref}
\usepackage{latexsym}

\def\name{Changxu Yan}

%pull in the necessary preamble matter for pygments output
\input{pygments.tex}

\parindent = 0.0 in
\parskip = 0.1 in

\begin{document}
\vfill
\begin{center}
    \textsc{\LARGE CS 444 Group 15}\\
    { \huge \bfseries Writing assignment 1 \\}
    \emph{Writer:}\\
    Changxu \textsc{YAN}
\end{center}

%input the pygmentized output of mt19937ar.c, using a (hopefully) unique name
%this file only exists at compile time. Feel free to change that.

\section{Processes}
\tab Overview:Blahbla \cite{Love}

this is 2 \cite{Wini}

\section{Threads}

\input{__example.c.tex}

\section{CPU scheduling}

\bibliographystyle{IEEEtran}
\bibliography{IEEEabrv,ref}
\end{document}
