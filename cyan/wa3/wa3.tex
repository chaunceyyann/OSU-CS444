\documentclass[10pt,draftclsnofoot,onecolumn,journal,compsoc]{IEEEtran}

\usepackage{graphicx}                                        
\usepackage{amssymb}                                         
\usepackage{amsmath}                                         
\usepackage{amsthm}                                          

\usepackage{alltt}                                           
\usepackage{float}
\usepackage{color}
\usepackage{url}

\usepackage{balance}
\usepackage[TABBOTCAP, tight]{subfigure}
\usepackage{enumitem}
\usepackage{pstricks, pst-node}

\usepackage{geometry}
\geometry{textheight=8.5in, textwidth=6in}
\usepackage{booktabs}
\usepackage{listings}
\usepackage{textcomp}
\usepackage{tikz}
\usetikzlibrary{shapes,arrows}
\usepackage{natbib}
\usepackage{indentfirst}

%random comment

\newcommand{\cred}[1]{{\color{red}#1}}
\newcommand{\cblue}[1]{{\color{blue}#1}}
\newcommand{\tab}{\hspace*{2em}} % for tabbing
\newcommand{\toc}{\tableofcontents}

\usepackage{hyperref}
\usepackage{latexsym}

\def\name{Changxu Yan}

%pull in the necessary preamble matter for pygments output
\usepackage{fancyvrb}
\usepackage{color}
\usepackage[latin1]{inputenc}


\makeatletter
\def\PY@reset{\let\PY@it=\relax \let\PY@bf=\relax%
    \let\PY@ul=\relax \let\PY@tc=\relax%
    \let\PY@bc=\relax \let\PY@ff=\relax}
\def\PY@tok#1{\csname PY@tok@#1\endcsname}
\def\PY@toks#1+{\ifx\relax#1\empty\else%
    \PY@tok{#1}\expandafter\PY@toks\fi}
\def\PY@do#1{\PY@bc{\PY@tc{\PY@ul{%
    \PY@it{\PY@bf{\PY@ff{#1}}}}}}}
\def\PY#1#2{\PY@reset\PY@toks#1+\relax+\PY@do{#2}}

\expandafter\def\csname PY@tok@gd\endcsname{\def\PY@tc##1{\textcolor[rgb]{0.63,0.00,0.00}{##1}}}
\expandafter\def\csname PY@tok@gu\endcsname{\let\PY@bf=\textbf\def\PY@tc##1{\textcolor[rgb]{0.50,0.00,0.50}{##1}}}
\expandafter\def\csname PY@tok@gt\endcsname{\def\PY@tc##1{\textcolor[rgb]{0.00,0.25,0.82}{##1}}}
\expandafter\def\csname PY@tok@gs\endcsname{\let\PY@bf=\textbf}
\expandafter\def\csname PY@tok@gr\endcsname{\def\PY@tc##1{\textcolor[rgb]{1.00,0.00,0.00}{##1}}}
\expandafter\def\csname PY@tok@cm\endcsname{\let\PY@it=\textit\def\PY@tc##1{\textcolor[rgb]{0.25,0.50,0.50}{##1}}}
\expandafter\def\csname PY@tok@vg\endcsname{\def\PY@tc##1{\textcolor[rgb]{0.10,0.09,0.49}{##1}}}
\expandafter\def\csname PY@tok@m\endcsname{\def\PY@tc##1{\textcolor[rgb]{0.40,0.40,0.40}{##1}}}
\expandafter\def\csname PY@tok@mh\endcsname{\def\PY@tc##1{\textcolor[rgb]{0.40,0.40,0.40}{##1}}}
\expandafter\def\csname PY@tok@go\endcsname{\def\PY@tc##1{\textcolor[rgb]{0.50,0.50,0.50}{##1}}}
\expandafter\def\csname PY@tok@ge\endcsname{\let\PY@it=\textit}
\expandafter\def\csname PY@tok@vc\endcsname{\def\PY@tc##1{\textcolor[rgb]{0.10,0.09,0.49}{##1}}}
\expandafter\def\csname PY@tok@il\endcsname{\def\PY@tc##1{\textcolor[rgb]{0.40,0.40,0.40}{##1}}}
\expandafter\def\csname PY@tok@cs\endcsname{\let\PY@it=\textit\def\PY@tc##1{\textcolor[rgb]{0.25,0.50,0.50}{##1}}}
\expandafter\def\csname PY@tok@cp\endcsname{\def\PY@tc##1{\textcolor[rgb]{0.74,0.48,0.00}{##1}}}
\expandafter\def\csname PY@tok@gi\endcsname{\def\PY@tc##1{\textcolor[rgb]{0.00,0.63,0.00}{##1}}}
\expandafter\def\csname PY@tok@gh\endcsname{\let\PY@bf=\textbf\def\PY@tc##1{\textcolor[rgb]{0.00,0.00,0.50}{##1}}}
\expandafter\def\csname PY@tok@ni\endcsname{\let\PY@bf=\textbf\def\PY@tc##1{\textcolor[rgb]{0.60,0.60,0.60}{##1}}}
\expandafter\def\csname PY@tok@nl\endcsname{\def\PY@tc##1{\textcolor[rgb]{0.63,0.63,0.00}{##1}}}
\expandafter\def\csname PY@tok@nn\endcsname{\let\PY@bf=\textbf\def\PY@tc##1{\textcolor[rgb]{0.00,0.00,1.00}{##1}}}
\expandafter\def\csname PY@tok@no\endcsname{\def\PY@tc##1{\textcolor[rgb]{0.53,0.00,0.00}{##1}}}
\expandafter\def\csname PY@tok@na\endcsname{\def\PY@tc##1{\textcolor[rgb]{0.49,0.56,0.16}{##1}}}
\expandafter\def\csname PY@tok@nb\endcsname{\def\PY@tc##1{\textcolor[rgb]{0.00,0.50,0.00}{##1}}}
\expandafter\def\csname PY@tok@nc\endcsname{\let\PY@bf=\textbf\def\PY@tc##1{\textcolor[rgb]{0.00,0.00,1.00}{##1}}}
\expandafter\def\csname PY@tok@nd\endcsname{\def\PY@tc##1{\textcolor[rgb]{0.67,0.13,1.00}{##1}}}
\expandafter\def\csname PY@tok@ne\endcsname{\let\PY@bf=\textbf\def\PY@tc##1{\textcolor[rgb]{0.82,0.25,0.23}{##1}}}
\expandafter\def\csname PY@tok@nf\endcsname{\def\PY@tc##1{\textcolor[rgb]{0.00,0.00,1.00}{##1}}}
\expandafter\def\csname PY@tok@si\endcsname{\let\PY@bf=\textbf\def\PY@tc##1{\textcolor[rgb]{0.73,0.40,0.53}{##1}}}
\expandafter\def\csname PY@tok@s2\endcsname{\def\PY@tc##1{\textcolor[rgb]{0.73,0.13,0.13}{##1}}}
\expandafter\def\csname PY@tok@vi\endcsname{\def\PY@tc##1{\textcolor[rgb]{0.10,0.09,0.49}{##1}}}
\expandafter\def\csname PY@tok@nt\endcsname{\let\PY@bf=\textbf\def\PY@tc##1{\textcolor[rgb]{0.00,0.50,0.00}{##1}}}
\expandafter\def\csname PY@tok@nv\endcsname{\def\PY@tc##1{\textcolor[rgb]{0.10,0.09,0.49}{##1}}}
\expandafter\def\csname PY@tok@s1\endcsname{\def\PY@tc##1{\textcolor[rgb]{0.73,0.13,0.13}{##1}}}
\expandafter\def\csname PY@tok@sh\endcsname{\def\PY@tc##1{\textcolor[rgb]{0.73,0.13,0.13}{##1}}}
\expandafter\def\csname PY@tok@sc\endcsname{\def\PY@tc##1{\textcolor[rgb]{0.73,0.13,0.13}{##1}}}
\expandafter\def\csname PY@tok@sx\endcsname{\def\PY@tc##1{\textcolor[rgb]{0.00,0.50,0.00}{##1}}}
\expandafter\def\csname PY@tok@bp\endcsname{\def\PY@tc##1{\textcolor[rgb]{0.00,0.50,0.00}{##1}}}
\expandafter\def\csname PY@tok@c1\endcsname{\let\PY@it=\textit\def\PY@tc##1{\textcolor[rgb]{0.25,0.50,0.50}{##1}}}
\expandafter\def\csname PY@tok@kc\endcsname{\let\PY@bf=\textbf\def\PY@tc##1{\textcolor[rgb]{0.00,0.50,0.00}{##1}}}
\expandafter\def\csname PY@tok@c\endcsname{\let\PY@it=\textit\def\PY@tc##1{\textcolor[rgb]{0.25,0.50,0.50}{##1}}}
\expandafter\def\csname PY@tok@mf\endcsname{\def\PY@tc##1{\textcolor[rgb]{0.40,0.40,0.40}{##1}}}
\expandafter\def\csname PY@tok@err\endcsname{\def\PY@bc##1{\setlength{\fboxsep}{0pt}\fcolorbox[rgb]{1.00,0.00,0.00}{1,1,1}{\strut ##1}}}
\expandafter\def\csname PY@tok@kd\endcsname{\let\PY@bf=\textbf\def\PY@tc##1{\textcolor[rgb]{0.00,0.50,0.00}{##1}}}
\expandafter\def\csname PY@tok@ss\endcsname{\def\PY@tc##1{\textcolor[rgb]{0.10,0.09,0.49}{##1}}}
\expandafter\def\csname PY@tok@sr\endcsname{\def\PY@tc##1{\textcolor[rgb]{0.73,0.40,0.53}{##1}}}
\expandafter\def\csname PY@tok@mo\endcsname{\def\PY@tc##1{\textcolor[rgb]{0.40,0.40,0.40}{##1}}}
\expandafter\def\csname PY@tok@kn\endcsname{\let\PY@bf=\textbf\def\PY@tc##1{\textcolor[rgb]{0.00,0.50,0.00}{##1}}}
\expandafter\def\csname PY@tok@mi\endcsname{\def\PY@tc##1{\textcolor[rgb]{0.40,0.40,0.40}{##1}}}
\expandafter\def\csname PY@tok@gp\endcsname{\let\PY@bf=\textbf\def\PY@tc##1{\textcolor[rgb]{0.00,0.00,0.50}{##1}}}
\expandafter\def\csname PY@tok@o\endcsname{\def\PY@tc##1{\textcolor[rgb]{0.40,0.40,0.40}{##1}}}
\expandafter\def\csname PY@tok@kr\endcsname{\let\PY@bf=\textbf\def\PY@tc##1{\textcolor[rgb]{0.00,0.50,0.00}{##1}}}
\expandafter\def\csname PY@tok@s\endcsname{\def\PY@tc##1{\textcolor[rgb]{0.73,0.13,0.13}{##1}}}
\expandafter\def\csname PY@tok@kp\endcsname{\def\PY@tc##1{\textcolor[rgb]{0.00,0.50,0.00}{##1}}}
\expandafter\def\csname PY@tok@w\endcsname{\def\PY@tc##1{\textcolor[rgb]{0.73,0.73,0.73}{##1}}}
\expandafter\def\csname PY@tok@kt\endcsname{\def\PY@tc##1{\textcolor[rgb]{0.69,0.00,0.25}{##1}}}
\expandafter\def\csname PY@tok@ow\endcsname{\let\PY@bf=\textbf\def\PY@tc##1{\textcolor[rgb]{0.67,0.13,1.00}{##1}}}
\expandafter\def\csname PY@tok@sb\endcsname{\def\PY@tc##1{\textcolor[rgb]{0.73,0.13,0.13}{##1}}}
\expandafter\def\csname PY@tok@k\endcsname{\let\PY@bf=\textbf\def\PY@tc##1{\textcolor[rgb]{0.00,0.50,0.00}{##1}}}
\expandafter\def\csname PY@tok@se\endcsname{\let\PY@bf=\textbf\def\PY@tc##1{\textcolor[rgb]{0.73,0.40,0.13}{##1}}}
\expandafter\def\csname PY@tok@sd\endcsname{\let\PY@it=\textit\def\PY@tc##1{\textcolor[rgb]{0.73,0.13,0.13}{##1}}}

\def\PYZbs{\char`\\}
\def\PYZus{\char`\_}
\def\PYZob{\char`\{}
\def\PYZcb{\char`\}}
\def\PYZca{\char`\^}
\def\PYZam{\char`\&}
\def\PYZlt{\char`\<}
\def\PYZgt{\char`\>}
\def\PYZsh{\char`\#}
\def\PYZpc{\char`\%}
\def\PYZdl{\char`\$}
\def\PYZti{\char`\~}
% for compatibility with earlier versions
\def\PYZat{@}
\def\PYZlb{[}
\def\PYZrb{]}
\makeatother


\parindent = 0.0 in
\parskip = 0.1 in

\begin{document}
\begin{titlepage}
    \begin{center}
        \vfill
        \textsc{\LARGE CS 444 Group 15}\par
        \vspace{1cm}
        { \huge \bfseries Writing assignment 3 \par}
        \vfill
        \emph{Writer:}\par
        Changxu \textsc{YAN}\par
        \vfill
        {\large \today\par}
    \end{center}
\end{titlepage}


\section{File Systems}
File systems are an integral part of any operating system. They allow users to upload and store files, provide access to data, and make hard drives useful. Different operating systems differ in their native file system. Modern Linux uses ext4, Windows uses NTFS, last but not least, FreeBSD uses ZFS. File systems are the fundamental principle behind how data is transferred in I/O situations. Every file store in a computer is stored and retrieved via a file system. Without a file system all of the data on the disk would be a pile of bytes unknowing where one file ends and another begins. For each file system there needs to be a way for the data structures to be written and read from the physical volume. This is where the file system drivers come in. Microsoft Windows has a very different way of approaching file systems than the Linux implementation. FreeBSD using \textit{ZFS} as their native file system which have been consider "next-gen file system" by folks. Linux, Windows and FreeBSD have the similar architecture but the way they are implemented with in the operating systems are very different.

\subsection*{Linux} 
Linux supports dozens of different file systems with the most common being Minux, Ext[2,3,4] and JFS. Linux has one of the most versatile file system support all because of the \textit{VFS} (Virtual File System). The \textit{VFS} creates the abstraction from real underlying file I/O operations. With all of the I/O operations flowing through the \textit{VFS} for Linux to support different file system all that is needed is a driver to translates standardized \textit{VFS} data operations to the current file system operations. This allows the Linux system to perform I/O operations without even needing to know what the actual file system of the disk is.~\cite{Love}.

The \textit{VFS} describes the system's files in terms of superblocks and inodes in much the same way as the EXT2 file system uses superblocks and inodes. Like the EXT2 inodes, the \textit{VFS} inodes describe files and directories within the system; the contents and topology of the Virtual File System. A \textit{VFS} specifies an interface between the kernel and a concrete file system. Therefore, it is easy to add support for new file system types to the kernel simply by fulfilling the contract. 

\subsection*{Windows} 
Windows on the other hand only supports a handful of file systems. Since Microsoft does not take advantage of a standardize virtual file system like Linux the operating system must know each file system and how to specifically talk to the media. Windows achieves this with an advanced \textit{I/O Manager} that in turn uses drivers just like the Linux kernel to perform the actually operations on to the file system~\cite{Wini}. 

Windows either at boot or when a new volume is mounted first tries to identify what type of filesytem the volume contains. Every Widows supported file system contains the information need for the \textit{FSD} (File system Driver) in the volumes boot sector. If the file system format is unrecognized or boot sector has been corrupt rendering the information Windows uses to identify the volumes file system unreadable Windows will default to the \textit{RAW} \textit{FSD}. When Windows declares a file system as \textit{RAW} the user is prompted to see if they would like to format the volume. If the boot sector is intact and Windows identifies the file system as a supported type Windows creates a \textit{Device Object} that the operating system will use to map the requested I/O operations to the physical media. After a \textit{FSD} claims a volume all of the I/O operations to that volume is passed trough the \textit{FSD}. All of the I/O operations are mapped from the \textit{Device Object} by the \textit{VPB} (Volume Parameter Block) to the volumes responsible \textit{FSD}. When Windows mounts a volume and assigns a drive letter to it, it is really just a symbolic link to the \textit{Device Object}~\cite{Wini}.

\subsection*{FreeBSD}
Traditionally, the native FreeBSD file system has been the Unix File System UFS which has been modernized as UFS2. Since FreeBSD 7.0, the \textit{Z File System (ZFS)} is also available as a native file system. FreeBSD also supports a multitude of other file systems so that data from other operating systems can be accessed locally, such as data stored on locally attached USB storage devices, flash drives, and hard disks. This includes support for the \textit{Linux® Extended File System (EXT)} and the Reiser file system.

\textit{ZFS} is significantly different from any previous file system because it is more than just a file system. Combining the traditionally separate roles of volume manager and file system provides \textit{ZFS} with unique advantages. Such as volume management, snapshots, checksumming, compression and deduplication, replication and more. The file system is now aware of the underlying structure of the disks. Traditional file systems could only be created on a single disk at a time. If there were two disks then two separate file systems would have to be created. In a traditional hardware RAID configuration, this problem was avoided by presenting the operating system with a single logical disk made up of the space provided by a number of physical disks, on top of which the operating system placed a file system. Even in the case of software RAID solutions like those provided by GEOM, the UFS file system living on top of the RAID transform believed that it was dealing with a single device. \textit{ZFS}'s combination of the volume manager and the file system solves this and allows the creation of many file systems all sharing a pool of available storage. One of the biggest advantages to \textit{ZFS}'s awareness of the physical layout of the disks is that existing file systems can be grown automatically when additional disks are added to the pool. This new space is then made available to all of the file systems. \textit{ZFS} also has a number of different properties that can be applied to each file system, giving many advantages to creating a number of different file systems and data sets rather than a single monolithic file system. \cite{BSDhandbook}

\section*{Comparison}
\noindent \textbf{Similarities}\\
\begin{itemize}
    \item Read information from the MBR(master Boot Record).
    \item Each file system gets a file descriptor mapped to unique location.
    \item Need a file system driver to write to actual volume.
\end{itemize}
\noindent \textbf{Differences}\\
\begin{itemize}
    \item Windows kernel needs to be aware of what the underlying file system is.
    \item Volumes in Windows get assigned letters, Volumes are assigned devices with numbers in Linux.
    \item FreeBSD ZFS utilizing virtual devices to manage one ore more disks like a RAID hardware controller do.
    \item FreeBSD ZFS using 256-bit checksum to ensure data integrity and reliability. 
\end{itemize}

\newpage
\bibliographystyle{IEEEtran}
\bibliography{ref}
\end{document}
