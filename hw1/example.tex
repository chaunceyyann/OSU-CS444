\documentclass[letterpaper,10pt]{article}

\usepackage{graphicx}                                        
\usepackage{amssymb}                                         
\usepackage{amsmath}                                         
\usepackage{amsthm}                                          

\usepackage{alltt}                                           
\usepackage{float}
\usepackage{color}
\usepackage{url}

\usepackage{balance}
\usepackage[TABBOTCAP, tight]{subfigure}
\usepackage{enumitem}
\usepackage{pstricks, pst-node}

\usepackage{geometry}
\geometry{textheight=8.5in, textwidth=6in}

%random comment

\newcommand{\cred}[1]{{\color{red}#1}}
\newcommand{\cblue}[1]{{\color{blue}#1}}

\newcommand{\toc}{\tableofcontents}

%\usepackage{hyperref}

\def\name{D. Kevin McGrath}

%pull in the necessary preamble matter for pygments output
\usepackage{fancyvrb}
\usepackage{color}
\usepackage[latin1]{inputenc}


\makeatletter
\def\PY@reset{\let\PY@it=\relax \let\PY@bf=\relax%
    \let\PY@ul=\relax \let\PY@tc=\relax%
    \let\PY@bc=\relax \let\PY@ff=\relax}
\def\PY@tok#1{\csname PY@tok@#1\endcsname}
\def\PY@toks#1+{\ifx\relax#1\empty\else%
    \PY@tok{#1}\expandafter\PY@toks\fi}
\def\PY@do#1{\PY@bc{\PY@tc{\PY@ul{%
    \PY@it{\PY@bf{\PY@ff{#1}}}}}}}
\def\PY#1#2{\PY@reset\PY@toks#1+\relax+\PY@do{#2}}

\expandafter\def\csname PY@tok@gd\endcsname{\def\PY@tc##1{\textcolor[rgb]{0.63,0.00,0.00}{##1}}}
\expandafter\def\csname PY@tok@gu\endcsname{\let\PY@bf=\textbf\def\PY@tc##1{\textcolor[rgb]{0.50,0.00,0.50}{##1}}}
\expandafter\def\csname PY@tok@gt\endcsname{\def\PY@tc##1{\textcolor[rgb]{0.00,0.25,0.82}{##1}}}
\expandafter\def\csname PY@tok@gs\endcsname{\let\PY@bf=\textbf}
\expandafter\def\csname PY@tok@gr\endcsname{\def\PY@tc##1{\textcolor[rgb]{1.00,0.00,0.00}{##1}}}
\expandafter\def\csname PY@tok@cm\endcsname{\let\PY@it=\textit\def\PY@tc##1{\textcolor[rgb]{0.25,0.50,0.50}{##1}}}
\expandafter\def\csname PY@tok@vg\endcsname{\def\PY@tc##1{\textcolor[rgb]{0.10,0.09,0.49}{##1}}}
\expandafter\def\csname PY@tok@m\endcsname{\def\PY@tc##1{\textcolor[rgb]{0.40,0.40,0.40}{##1}}}
\expandafter\def\csname PY@tok@mh\endcsname{\def\PY@tc##1{\textcolor[rgb]{0.40,0.40,0.40}{##1}}}
\expandafter\def\csname PY@tok@go\endcsname{\def\PY@tc##1{\textcolor[rgb]{0.50,0.50,0.50}{##1}}}
\expandafter\def\csname PY@tok@ge\endcsname{\let\PY@it=\textit}
\expandafter\def\csname PY@tok@vc\endcsname{\def\PY@tc##1{\textcolor[rgb]{0.10,0.09,0.49}{##1}}}
\expandafter\def\csname PY@tok@il\endcsname{\def\PY@tc##1{\textcolor[rgb]{0.40,0.40,0.40}{##1}}}
\expandafter\def\csname PY@tok@cs\endcsname{\let\PY@it=\textit\def\PY@tc##1{\textcolor[rgb]{0.25,0.50,0.50}{##1}}}
\expandafter\def\csname PY@tok@cp\endcsname{\def\PY@tc##1{\textcolor[rgb]{0.74,0.48,0.00}{##1}}}
\expandafter\def\csname PY@tok@gi\endcsname{\def\PY@tc##1{\textcolor[rgb]{0.00,0.63,0.00}{##1}}}
\expandafter\def\csname PY@tok@gh\endcsname{\let\PY@bf=\textbf\def\PY@tc##1{\textcolor[rgb]{0.00,0.00,0.50}{##1}}}
\expandafter\def\csname PY@tok@ni\endcsname{\let\PY@bf=\textbf\def\PY@tc##1{\textcolor[rgb]{0.60,0.60,0.60}{##1}}}
\expandafter\def\csname PY@tok@nl\endcsname{\def\PY@tc##1{\textcolor[rgb]{0.63,0.63,0.00}{##1}}}
\expandafter\def\csname PY@tok@nn\endcsname{\let\PY@bf=\textbf\def\PY@tc##1{\textcolor[rgb]{0.00,0.00,1.00}{##1}}}
\expandafter\def\csname PY@tok@no\endcsname{\def\PY@tc##1{\textcolor[rgb]{0.53,0.00,0.00}{##1}}}
\expandafter\def\csname PY@tok@na\endcsname{\def\PY@tc##1{\textcolor[rgb]{0.49,0.56,0.16}{##1}}}
\expandafter\def\csname PY@tok@nb\endcsname{\def\PY@tc##1{\textcolor[rgb]{0.00,0.50,0.00}{##1}}}
\expandafter\def\csname PY@tok@nc\endcsname{\let\PY@bf=\textbf\def\PY@tc##1{\textcolor[rgb]{0.00,0.00,1.00}{##1}}}
\expandafter\def\csname PY@tok@nd\endcsname{\def\PY@tc##1{\textcolor[rgb]{0.67,0.13,1.00}{##1}}}
\expandafter\def\csname PY@tok@ne\endcsname{\let\PY@bf=\textbf\def\PY@tc##1{\textcolor[rgb]{0.82,0.25,0.23}{##1}}}
\expandafter\def\csname PY@tok@nf\endcsname{\def\PY@tc##1{\textcolor[rgb]{0.00,0.00,1.00}{##1}}}
\expandafter\def\csname PY@tok@si\endcsname{\let\PY@bf=\textbf\def\PY@tc##1{\textcolor[rgb]{0.73,0.40,0.53}{##1}}}
\expandafter\def\csname PY@tok@s2\endcsname{\def\PY@tc##1{\textcolor[rgb]{0.73,0.13,0.13}{##1}}}
\expandafter\def\csname PY@tok@vi\endcsname{\def\PY@tc##1{\textcolor[rgb]{0.10,0.09,0.49}{##1}}}
\expandafter\def\csname PY@tok@nt\endcsname{\let\PY@bf=\textbf\def\PY@tc##1{\textcolor[rgb]{0.00,0.50,0.00}{##1}}}
\expandafter\def\csname PY@tok@nv\endcsname{\def\PY@tc##1{\textcolor[rgb]{0.10,0.09,0.49}{##1}}}
\expandafter\def\csname PY@tok@s1\endcsname{\def\PY@tc##1{\textcolor[rgb]{0.73,0.13,0.13}{##1}}}
\expandafter\def\csname PY@tok@sh\endcsname{\def\PY@tc##1{\textcolor[rgb]{0.73,0.13,0.13}{##1}}}
\expandafter\def\csname PY@tok@sc\endcsname{\def\PY@tc##1{\textcolor[rgb]{0.73,0.13,0.13}{##1}}}
\expandafter\def\csname PY@tok@sx\endcsname{\def\PY@tc##1{\textcolor[rgb]{0.00,0.50,0.00}{##1}}}
\expandafter\def\csname PY@tok@bp\endcsname{\def\PY@tc##1{\textcolor[rgb]{0.00,0.50,0.00}{##1}}}
\expandafter\def\csname PY@tok@c1\endcsname{\let\PY@it=\textit\def\PY@tc##1{\textcolor[rgb]{0.25,0.50,0.50}{##1}}}
\expandafter\def\csname PY@tok@kc\endcsname{\let\PY@bf=\textbf\def\PY@tc##1{\textcolor[rgb]{0.00,0.50,0.00}{##1}}}
\expandafter\def\csname PY@tok@c\endcsname{\let\PY@it=\textit\def\PY@tc##1{\textcolor[rgb]{0.25,0.50,0.50}{##1}}}
\expandafter\def\csname PY@tok@mf\endcsname{\def\PY@tc##1{\textcolor[rgb]{0.40,0.40,0.40}{##1}}}
\expandafter\def\csname PY@tok@err\endcsname{\def\PY@bc##1{\setlength{\fboxsep}{0pt}\fcolorbox[rgb]{1.00,0.00,0.00}{1,1,1}{\strut ##1}}}
\expandafter\def\csname PY@tok@kd\endcsname{\let\PY@bf=\textbf\def\PY@tc##1{\textcolor[rgb]{0.00,0.50,0.00}{##1}}}
\expandafter\def\csname PY@tok@ss\endcsname{\def\PY@tc##1{\textcolor[rgb]{0.10,0.09,0.49}{##1}}}
\expandafter\def\csname PY@tok@sr\endcsname{\def\PY@tc##1{\textcolor[rgb]{0.73,0.40,0.53}{##1}}}
\expandafter\def\csname PY@tok@mo\endcsname{\def\PY@tc##1{\textcolor[rgb]{0.40,0.40,0.40}{##1}}}
\expandafter\def\csname PY@tok@kn\endcsname{\let\PY@bf=\textbf\def\PY@tc##1{\textcolor[rgb]{0.00,0.50,0.00}{##1}}}
\expandafter\def\csname PY@tok@mi\endcsname{\def\PY@tc##1{\textcolor[rgb]{0.40,0.40,0.40}{##1}}}
\expandafter\def\csname PY@tok@gp\endcsname{\let\PY@bf=\textbf\def\PY@tc##1{\textcolor[rgb]{0.00,0.00,0.50}{##1}}}
\expandafter\def\csname PY@tok@o\endcsname{\def\PY@tc##1{\textcolor[rgb]{0.40,0.40,0.40}{##1}}}
\expandafter\def\csname PY@tok@kr\endcsname{\let\PY@bf=\textbf\def\PY@tc##1{\textcolor[rgb]{0.00,0.50,0.00}{##1}}}
\expandafter\def\csname PY@tok@s\endcsname{\def\PY@tc##1{\textcolor[rgb]{0.73,0.13,0.13}{##1}}}
\expandafter\def\csname PY@tok@kp\endcsname{\def\PY@tc##1{\textcolor[rgb]{0.00,0.50,0.00}{##1}}}
\expandafter\def\csname PY@tok@w\endcsname{\def\PY@tc##1{\textcolor[rgb]{0.73,0.73,0.73}{##1}}}
\expandafter\def\csname PY@tok@kt\endcsname{\def\PY@tc##1{\textcolor[rgb]{0.69,0.00,0.25}{##1}}}
\expandafter\def\csname PY@tok@ow\endcsname{\let\PY@bf=\textbf\def\PY@tc##1{\textcolor[rgb]{0.67,0.13,1.00}{##1}}}
\expandafter\def\csname PY@tok@sb\endcsname{\def\PY@tc##1{\textcolor[rgb]{0.73,0.13,0.13}{##1}}}
\expandafter\def\csname PY@tok@k\endcsname{\let\PY@bf=\textbf\def\PY@tc##1{\textcolor[rgb]{0.00,0.50,0.00}{##1}}}
\expandafter\def\csname PY@tok@se\endcsname{\let\PY@bf=\textbf\def\PY@tc##1{\textcolor[rgb]{0.73,0.40,0.13}{##1}}}
\expandafter\def\csname PY@tok@sd\endcsname{\let\PY@it=\textit\def\PY@tc##1{\textcolor[rgb]{0.73,0.13,0.13}{##1}}}

\def\PYZbs{\char`\\}
\def\PYZus{\char`\_}
\def\PYZob{\char`\{}
\def\PYZcb{\char`\}}
\def\PYZca{\char`\^}
\def\PYZam{\char`\&}
\def\PYZlt{\char`\<}
\def\PYZgt{\char`\>}
\def\PYZsh{\char`\#}
\def\PYZpc{\char`\%}
\def\PYZdl{\char`\$}
\def\PYZti{\char`\~}
% for compatibility with earlier versions
\def\PYZat{@}
\def\PYZlb{[}
\def\PYZrb{]}
\makeatother


%% The following metadata will show up in the PDF properties
% \hypersetup{
%   colorlinks = false,
%   urlcolor = black,
%   pdfauthor = {\name},
%   pdfkeywords = {cs311 ``operating systems'' files filesystem I/O},
%   pdftitle = {CS 311 Project 1: UNIX File I/O},
%   pdfsubject = {CS 311 Project 1},
%   pdfpagemode = UseNone
% }

\parindent = 0.0 in
\parskip = 0.1 in

\begin{document}

\tableofcontents

%input the pygmentized output of mt19937ar.c, using a (hopefully) unique name
%this file only exists at compile time. Feel free to change that.


\section{Section 1}
\subsection{blah}
\subsubsection{yada yada}
This is a paragraph in \LaTeX.

This is a new paragraph.

\begin{itemize}
\item \begin{equation}
    \label{eq1}
    \int_0^\pi \sin(x) \partial x
    \end{equation}
\item $\backslash$ As seen in Eq. \ref{eq1}, blah blah blah
\end{itemize}

\emph{\textbf{\color{red}This is italicized and red}}

\section*{Appendix 1: Source Code}
%\begin{Verbatim}[commandchars=\\\{\}]
\PY{c+cm}{/* }
\PY{c+cm}{   A C-program for MT19937, with initialization improved 2002/1/26.}
\PY{c+cm}{   Coded by Takuji Nishimura and Makoto Matsumoto.}

\PY{c+cm}{   Before using, initialize the state by using init\PYZus{}genrand(seed)  }
\PY{c+cm}{   or init\PYZus{}by\PYZus{}array(init\PYZus{}key, key\PYZus{}length).}

\PY{c+cm}{   Copyright (C) 1997 - 2002, Makoto Matsumoto and Takuji Nishimura,}
\PY{c+cm}{   All rights reserved.                          }

\PY{c+cm}{   Redistribution and use in source and binary forms, with or without}
\PY{c+cm}{   modification, are permitted provided that the following conditions}
\PY{c+cm}{   are met:}

\PY{c+cm}{   1. Redistributions of source code must retain the above copyright}
\PY{c+cm}{   notice, this list of conditions and the following disclaimer.}

\PY{c+cm}{   2. Redistributions in binary form must reproduce the above copyright}
\PY{c+cm}{   notice, this list of conditions and the following disclaimer in the}
\PY{c+cm}{   documentation and/or other materials provided with the distribution.}

\PY{c+cm}{   3. The names of its contributors may not be used to endorse or promote }
\PY{c+cm}{   products derived from this software without specific prior written }
\PY{c+cm}{   permission.}

\PY{c+cm}{   THIS SOFTWARE IS PROVIDED BY THE COPYRIGHT HOLDERS AND CONTRIBUTORS}
\PY{c+cm}{   "AS IS" AND ANY EXPRESS OR IMPLIED WARRANTIES, INCLUDING, BUT NOT}
\PY{c+cm}{   LIMITED TO, THE IMPLIED WARRANTIES OF MERCHANTABILITY AND FITNESS FOR}
\PY{c+cm}{   A PARTICULAR PURPOSE ARE DISCLAIMED.  IN NO EVENT SHALL THE COPYRIGHT OWNER OR}
\PY{c+cm}{   CONTRIBUTORS BE LIABLE FOR ANY DIRECT, INDIRECT, INCIDENTAL, SPECIAL,}
\PY{c+cm}{   EXEMPLARY, OR CONSEQUENTIAL DAMAGES (INCLUDING, BUT NOT LIMITED TO,}
\PY{c+cm}{   PROCUREMENT OF SUBSTITUTE GOODS OR SERVICES; LOSS OF USE, DATA, OR}
\PY{c+cm}{   PROFITS; OR BUSINESS INTERRUPTION) HOWEVER CAUSED AND ON ANY THEORY OF}
\PY{c+cm}{   LIABILITY, WHETHER IN CONTRACT, STRICT LIABILITY, OR TORT (INCLUDING}
\PY{c+cm}{   NEGLIGENCE OR OTHERWISE) ARISING IN ANY WAY OUT OF THE USE OF THIS}
\PY{c+cm}{   SOFTWARE, EVEN IF ADVISED OF THE POSSIBILITY OF SUCH DAMAGE.}


\PY{c+cm}{   Any feedback is very welcome.}
\PY{c+cm}{   http://www.math.sci.hiroshima-u.ac.jp/\PYZti{}m-mat/MT/emt.html}
\PY{c+cm}{   email: m-mat @ math.sci.hiroshima-u.ac.jp (remove space)}
\PY{c+cm}{*/}

\PY{c+cp}{\PYZsh{}}\PY{c+cp}{include <stdio.h>}

\PY{c+cp}{/* Period parameters */  }
\PY{c+cp}{\PYZsh{}}\PY{c+cp}{define N 624}
\PY{c+cp}{\PYZsh{}}\PY{c+cp}{define M 397}
\PY{c+cp}{\PYZsh{}}\PY{c+cp}{define MATRIX\PYZus{}A 0x9908b0dfUL   }\PY{c+cm}{/* constant vector a */}
\PY{c+cp}{\PYZsh{}}\PY{c+cp}{define UPPER\PYZus{}MASK 0x80000000UL }\PY{c+cm}{/* most significant w-r bits */}
\PY{c+cp}{\PYZsh{}}\PY{c+cp}{define LOWER\PYZus{}MASK 0x7fffffffUL }\PY{c+cm}{/* least significant r bits */}

\PY{k}{static} \PY{k+kt}{unsigned} \PY{k+kt}{long} \PY{n}{mt}\PY{p}{[}\PY{n}{N}\PY{p}{]}\PY{p}{;} \PY{c+cm}{/* the array for the state vector  */}
\PY{k}{static} \PY{k+kt}{int} \PY{n}{mti}\PY{o}{=}\PY{n}{N}\PY{o}{+}\PY{l+m+mi}{1}\PY{p}{;} \PY{c+cm}{/* mti==N+1 means mt[N] is not initialized */}

\PY{c+cm}{/* initializes mt[N] with a seed */}
\PY{k+kt}{void} \PY{n+nf}{init\PYZus{}genrand}\PY{p}{(}\PY{k+kt}{unsigned} \PY{k+kt}{long} \PY{n}{s}\PY{p}{)}
\PY{p}{\PYZob{}}
	\PY{n}{mt}\PY{p}{[}\PY{l+m+mi}{0}\PY{p}{]}\PY{o}{=} \PY{n}{s} \PY{o}{&} \PY{l+m+mh}{0xffffffffUL}\PY{p}{;}
	\PY{k}{for} \PY{p}{(}\PY{n}{mti}\PY{o}{=}\PY{l+m+mi}{1}\PY{p}{;} \PY{n}{mti}\PY{o}{<}\PY{n}{N}\PY{p}{;} \PY{n}{mti}\PY{o}{+}\PY{o}{+}\PY{p}{)} \PY{p}{\PYZob{}}
		\PY{n}{mt}\PY{p}{[}\PY{n}{mti}\PY{p}{]} \PY{o}{=} 
			\PY{p}{(}\PY{l+m+mi}{1812433253UL} \PY{o}{*} \PY{p}{(}\PY{n}{mt}\PY{p}{[}\PY{n}{mti}\PY{o}{-}\PY{l+m+mi}{1}\PY{p}{]} \PY{o}{\PYZca{}} \PY{p}{(}\PY{n}{mt}\PY{p}{[}\PY{n}{mti}\PY{o}{-}\PY{l+m+mi}{1}\PY{p}{]} \PY{o}{>}\PY{o}{>} \PY{l+m+mi}{30}\PY{p}{)}\PY{p}{)} \PY{o}{+} \PY{n}{mti}\PY{p}{)}\PY{p}{;} 
		\PY{c+cm}{/* See Knuth TAOCP Vol2. 3rd Ed. P.106 for multiplier. */}
		\PY{c+cm}{/* In the previous versions, MSBs of the seed affect   */}
		\PY{c+cm}{/* only MSBs of the array mt[].                        */}
		\PY{c+cm}{/* 2002/01/09 modified by Makoto Matsumoto             */}
		\PY{n}{mt}\PY{p}{[}\PY{n}{mti}\PY{p}{]} \PY{o}{&}\PY{o}{=} \PY{l+m+mh}{0xffffffffUL}\PY{p}{;}
		\PY{c+cm}{/* for >32 bit machines */}
	\PY{p}{\PYZcb{}}
\PY{p}{\PYZcb{}}

\PY{c+cm}{/* initialize by an array with array-length */}
\PY{c+cm}{/* init\PYZus{}key is the array for initializing keys */}
\PY{c+cm}{/* key\PYZus{}length is its length */}
\PY{c+cm}{/* slight change for C++, 2004/2/26 */}
\PY{k+kt}{void} \PY{n+nf}{init\PYZus{}by\PYZus{}array}\PY{p}{(}\PY{k+kt}{unsigned} \PY{k+kt}{long} \PY{n}{init\PYZus{}key}\PY{p}{[}\PY{p}{]}\PY{p}{,} \PY{k+kt}{int} \PY{n}{key\PYZus{}length}\PY{p}{)}
\PY{p}{\PYZob{}}
	\PY{k+kt}{int} \PY{n}{i}\PY{p}{,} \PY{n}{j}\PY{p}{,} \PY{n}{k}\PY{p}{;}
	\PY{n}{init\PYZus{}genrand}\PY{p}{(}\PY{l+m+mi}{19650218UL}\PY{p}{)}\PY{p}{;}
	\PY{n}{i}\PY{o}{=}\PY{l+m+mi}{1}\PY{p}{;} \PY{n}{j}\PY{o}{=}\PY{l+m+mi}{0}\PY{p}{;}
	\PY{n}{k} \PY{o}{=} \PY{p}{(}\PY{n}{N}\PY{o}{>}\PY{n}{key\PYZus{}length} \PY{o}{?} \PY{n}{N} \PY{o}{:} \PY{n}{key\PYZus{}length}\PY{p}{)}\PY{p}{;}
	\PY{k}{for} \PY{p}{(}\PY{p}{;} \PY{n}{k}\PY{p}{;} \PY{n}{k}\PY{o}{-}\PY{o}{-}\PY{p}{)} \PY{p}{\PYZob{}}
		\PY{n}{mt}\PY{p}{[}\PY{n}{i}\PY{p}{]} \PY{o}{=} \PY{p}{(}\PY{n}{mt}\PY{p}{[}\PY{n}{i}\PY{p}{]} \PY{o}{\PYZca{}} \PY{p}{(}\PY{p}{(}\PY{n}{mt}\PY{p}{[}\PY{n}{i}\PY{o}{-}\PY{l+m+mi}{1}\PY{p}{]} \PY{o}{\PYZca{}} \PY{p}{(}\PY{n}{mt}\PY{p}{[}\PY{n}{i}\PY{o}{-}\PY{l+m+mi}{1}\PY{p}{]} \PY{o}{>}\PY{o}{>} \PY{l+m+mi}{30}\PY{p}{)}\PY{p}{)} \PY{o}{*} \PY{l+m+mi}{1664525UL}\PY{p}{)}\PY{p}{)}
			\PY{o}{+} \PY{n}{init\PYZus{}key}\PY{p}{[}\PY{n}{j}\PY{p}{]} \PY{o}{+} \PY{n}{j}\PY{p}{;} \PY{c+cm}{/* non linear */}
		\PY{n}{mt}\PY{p}{[}\PY{n}{i}\PY{p}{]} \PY{o}{&}\PY{o}{=} \PY{l+m+mh}{0xffffffffUL}\PY{p}{;} \PY{c+cm}{/* for WORDSIZE > 32 machines */}
		\PY{n}{i}\PY{o}{+}\PY{o}{+}\PY{p}{;} \PY{n}{j}\PY{o}{+}\PY{o}{+}\PY{p}{;}
		\PY{k}{if} \PY{p}{(}\PY{n}{i}\PY{o}{>}\PY{o}{=}\PY{n}{N}\PY{p}{)} \PY{p}{\PYZob{}} \PY{n}{mt}\PY{p}{[}\PY{l+m+mi}{0}\PY{p}{]} \PY{o}{=} \PY{n}{mt}\PY{p}{[}\PY{n}{N}\PY{o}{-}\PY{l+m+mi}{1}\PY{p}{]}\PY{p}{;} \PY{n}{i}\PY{o}{=}\PY{l+m+mi}{1}\PY{p}{;} \PY{p}{\PYZcb{}}
		\PY{k}{if} \PY{p}{(}\PY{n}{j}\PY{o}{>}\PY{o}{=}\PY{n}{key\PYZus{}length}\PY{p}{)} \PY{n}{j}\PY{o}{=}\PY{l+m+mi}{0}\PY{p}{;}
	\PY{p}{\PYZcb{}}
	\PY{k}{for} \PY{p}{(}\PY{n}{k}\PY{o}{=}\PY{n}{N}\PY{o}{-}\PY{l+m+mi}{1}\PY{p}{;} \PY{n}{k}\PY{p}{;} \PY{n}{k}\PY{o}{-}\PY{o}{-}\PY{p}{)} \PY{p}{\PYZob{}}
		\PY{n}{mt}\PY{p}{[}\PY{n}{i}\PY{p}{]} \PY{o}{=} \PY{p}{(}\PY{n}{mt}\PY{p}{[}\PY{n}{i}\PY{p}{]} \PY{o}{\PYZca{}} \PY{p}{(}\PY{p}{(}\PY{n}{mt}\PY{p}{[}\PY{n}{i}\PY{o}{-}\PY{l+m+mi}{1}\PY{p}{]} \PY{o}{\PYZca{}} \PY{p}{(}\PY{n}{mt}\PY{p}{[}\PY{n}{i}\PY{o}{-}\PY{l+m+mi}{1}\PY{p}{]} \PY{o}{>}\PY{o}{>} \PY{l+m+mi}{30}\PY{p}{)}\PY{p}{)} \PY{o}{*} \PY{l+m+mi}{1566083941UL}\PY{p}{)}\PY{p}{)}
			\PY{o}{-} \PY{n}{i}\PY{p}{;} \PY{c+cm}{/* non linear */}
		\PY{n}{mt}\PY{p}{[}\PY{n}{i}\PY{p}{]} \PY{o}{&}\PY{o}{=} \PY{l+m+mh}{0xffffffffUL}\PY{p}{;} \PY{c+cm}{/* for WORDSIZE > 32 machines */}
		\PY{n}{i}\PY{o}{+}\PY{o}{+}\PY{p}{;}
		\PY{k}{if} \PY{p}{(}\PY{n}{i}\PY{o}{>}\PY{o}{=}\PY{n}{N}\PY{p}{)} \PY{p}{\PYZob{}} \PY{n}{mt}\PY{p}{[}\PY{l+m+mi}{0}\PY{p}{]} \PY{o}{=} \PY{n}{mt}\PY{p}{[}\PY{n}{N}\PY{o}{-}\PY{l+m+mi}{1}\PY{p}{]}\PY{p}{;} \PY{n}{i}\PY{o}{=}\PY{l+m+mi}{1}\PY{p}{;} \PY{p}{\PYZcb{}}
	\PY{p}{\PYZcb{}}

	\PY{n}{mt}\PY{p}{[}\PY{l+m+mi}{0}\PY{p}{]} \PY{o}{=} \PY{l+m+mh}{0x80000000UL}\PY{p}{;} \PY{c+cm}{/* MSB is 1; assuring non-zero initial array */} 
\PY{p}{\PYZcb{}}

\PY{c+cm}{/* generates a random number on [0,0xffffffff]-interval */}
\PY{k+kt}{unsigned} \PY{k+kt}{long} \PY{n+nf}{genrand\PYZus{}int32}\PY{p}{(}\PY{k+kt}{void}\PY{p}{)}
\PY{p}{\PYZob{}}
	\PY{k+kt}{unsigned} \PY{k+kt}{long} \PY{n}{y}\PY{p}{;}
	\PY{k}{static} \PY{k+kt}{unsigned} \PY{k+kt}{long} \PY{n}{mag01}\PY{p}{[}\PY{l+m+mi}{2}\PY{p}{]}\PY{o}{=}\PY{p}{\PYZob{}}\PY{l+m+mh}{0x0UL}\PY{p}{,} \PY{n}{MATRIX\PYZus{}A}\PY{p}{\PYZcb{}}\PY{p}{;}
	\PY{c+cm}{/* mag01[x] = x * MATRIX\PYZus{}A  for x=0,1 */}

	\PY{k}{if} \PY{p}{(}\PY{n}{mti} \PY{o}{>}\PY{o}{=} \PY{n}{N}\PY{p}{)} \PY{p}{\PYZob{}} \PY{c+cm}{/* generate N words at one time */}
		\PY{k+kt}{int} \PY{n}{kk}\PY{p}{;}

		\PY{k}{if} \PY{p}{(}\PY{n}{mti} \PY{o}{=}\PY{o}{=} \PY{n}{N}\PY{o}{+}\PY{l+m+mi}{1}\PY{p}{)}   \PY{c+cm}{/* if init\PYZus{}genrand() has not been called, */}
			\PY{n}{init\PYZus{}genrand}\PY{p}{(}\PY{l+m+mi}{5489UL}\PY{p}{)}\PY{p}{;} \PY{c+cm}{/* a default initial seed is used */}

		\PY{k}{for} \PY{p}{(}\PY{n}{kk}\PY{o}{=}\PY{l+m+mi}{0}\PY{p}{;}\PY{n}{kk}\PY{o}{<}\PY{n}{N}\PY{o}{-}\PY{n}{M}\PY{p}{;}\PY{n}{kk}\PY{o}{+}\PY{o}{+}\PY{p}{)} \PY{p}{\PYZob{}}
			\PY{n}{y} \PY{o}{=} \PY{p}{(}\PY{n}{mt}\PY{p}{[}\PY{n}{kk}\PY{p}{]}\PY{o}{&}\PY{n}{UPPER\PYZus{}MASK}\PY{p}{)}\PY{o}{|}\PY{p}{(}\PY{n}{mt}\PY{p}{[}\PY{n}{kk}\PY{o}{+}\PY{l+m+mi}{1}\PY{p}{]}\PY{o}{&}\PY{n}{LOWER\PYZus{}MASK}\PY{p}{)}\PY{p}{;}
			\PY{n}{mt}\PY{p}{[}\PY{n}{kk}\PY{p}{]} \PY{o}{=} \PY{n}{mt}\PY{p}{[}\PY{n}{kk}\PY{o}{+}\PY{n}{M}\PY{p}{]} \PY{o}{\PYZca{}} \PY{p}{(}\PY{n}{y} \PY{o}{>}\PY{o}{>} \PY{l+m+mi}{1}\PY{p}{)} \PY{o}{\PYZca{}} \PY{n}{mag01}\PY{p}{[}\PY{n}{y} \PY{o}{&} \PY{l+m+mh}{0x1UL}\PY{p}{]}\PY{p}{;}
		\PY{p}{\PYZcb{}}
		\PY{k}{for} \PY{p}{(}\PY{p}{;}\PY{n}{kk}\PY{o}{<}\PY{n}{N}\PY{o}{-}\PY{l+m+mi}{1}\PY{p}{;}\PY{n}{kk}\PY{o}{+}\PY{o}{+}\PY{p}{)} \PY{p}{\PYZob{}}
			\PY{n}{y} \PY{o}{=} \PY{p}{(}\PY{n}{mt}\PY{p}{[}\PY{n}{kk}\PY{p}{]}\PY{o}{&}\PY{n}{UPPER\PYZus{}MASK}\PY{p}{)}\PY{o}{|}\PY{p}{(}\PY{n}{mt}\PY{p}{[}\PY{n}{kk}\PY{o}{+}\PY{l+m+mi}{1}\PY{p}{]}\PY{o}{&}\PY{n}{LOWER\PYZus{}MASK}\PY{p}{)}\PY{p}{;}
			\PY{n}{mt}\PY{p}{[}\PY{n}{kk}\PY{p}{]} \PY{o}{=} \PY{n}{mt}\PY{p}{[}\PY{n}{kk}\PY{o}{+}\PY{p}{(}\PY{n}{M}\PY{o}{-}\PY{n}{N}\PY{p}{)}\PY{p}{]} \PY{o}{\PYZca{}} \PY{p}{(}\PY{n}{y} \PY{o}{>}\PY{o}{>} \PY{l+m+mi}{1}\PY{p}{)} \PY{o}{\PYZca{}} \PY{n}{mag01}\PY{p}{[}\PY{n}{y} \PY{o}{&} \PY{l+m+mh}{0x1UL}\PY{p}{]}\PY{p}{;}
		\PY{p}{\PYZcb{}}
		\PY{n}{y} \PY{o}{=} \PY{p}{(}\PY{n}{mt}\PY{p}{[}\PY{n}{N}\PY{o}{-}\PY{l+m+mi}{1}\PY{p}{]}\PY{o}{&}\PY{n}{UPPER\PYZus{}MASK}\PY{p}{)}\PY{o}{|}\PY{p}{(}\PY{n}{mt}\PY{p}{[}\PY{l+m+mi}{0}\PY{p}{]}\PY{o}{&}\PY{n}{LOWER\PYZus{}MASK}\PY{p}{)}\PY{p}{;}
		\PY{n}{mt}\PY{p}{[}\PY{n}{N}\PY{o}{-}\PY{l+m+mi}{1}\PY{p}{]} \PY{o}{=} \PY{n}{mt}\PY{p}{[}\PY{n}{M}\PY{o}{-}\PY{l+m+mi}{1}\PY{p}{]} \PY{o}{\PYZca{}} \PY{p}{(}\PY{n}{y} \PY{o}{>}\PY{o}{>} \PY{l+m+mi}{1}\PY{p}{)} \PY{o}{\PYZca{}} \PY{n}{mag01}\PY{p}{[}\PY{n}{y} \PY{o}{&} \PY{l+m+mh}{0x1UL}\PY{p}{]}\PY{p}{;}

		\PY{n}{mti} \PY{o}{=} \PY{l+m+mi}{0}\PY{p}{;}
	\PY{p}{\PYZcb{}}
  
	\PY{n}{y} \PY{o}{=} \PY{n}{mt}\PY{p}{[}\PY{n}{mti}\PY{o}{+}\PY{o}{+}\PY{p}{]}\PY{p}{;}

	\PY{c+cm}{/* Tempering */}
	\PY{n}{y} \PY{o}{\PYZca{}}\PY{o}{=} \PY{p}{(}\PY{n}{y} \PY{o}{>}\PY{o}{>} \PY{l+m+mi}{11}\PY{p}{)}\PY{p}{;}
	\PY{n}{y} \PY{o}{\PYZca{}}\PY{o}{=} \PY{p}{(}\PY{n}{y} \PY{o}{<}\PY{o}{<} \PY{l+m+mi}{7}\PY{p}{)} \PY{o}{&} \PY{l+m+mh}{0x9d2c5680UL}\PY{p}{;}
	\PY{n}{y} \PY{o}{\PYZca{}}\PY{o}{=} \PY{p}{(}\PY{n}{y} \PY{o}{<}\PY{o}{<} \PY{l+m+mi}{15}\PY{p}{)} \PY{o}{&} \PY{l+m+mh}{0xefc60000UL}\PY{p}{;}
	\PY{n}{y} \PY{o}{\PYZca{}}\PY{o}{=} \PY{p}{(}\PY{n}{y} \PY{o}{>}\PY{o}{>} \PY{l+m+mi}{18}\PY{p}{)}\PY{p}{;}

	\PY{k}{return} \PY{n}{y}\PY{p}{;}
\PY{p}{\PYZcb{}}

\PY{c+cm}{/* generates a random number on [0,0x7fffffff]-interval */}
\PY{k+kt}{long} \PY{n+nf}{genrand\PYZus{}int31}\PY{p}{(}\PY{k+kt}{void}\PY{p}{)}
\PY{p}{\PYZob{}}
	\PY{k}{return} \PY{p}{(}\PY{k+kt}{long}\PY{p}{)}\PY{p}{(}\PY{n}{genrand\PYZus{}int32}\PY{p}{(}\PY{p}{)}\PY{o}{>}\PY{o}{>}\PY{l+m+mi}{1}\PY{p}{)}\PY{p}{;}
\PY{p}{\PYZcb{}}

\PY{c+cm}{/* generates a random number on [0,1]-real-interval */}
\PY{k+kt}{double} \PY{n+nf}{genrand\PYZus{}real1}\PY{p}{(}\PY{k+kt}{void}\PY{p}{)}
\PY{p}{\PYZob{}}
	\PY{k}{return} \PY{n}{genrand\PYZus{}int32}\PY{p}{(}\PY{p}{)}\PY{o}{*}\PY{p}{(}\PY{l+m+mf}{1.0}\PY{o}{/}\PY{l+m+mf}{4294967295.0}\PY{p}{)}\PY{p}{;} 
	\PY{c+cm}{/* divided by 2\PYZca{}32-1 */} 
\PY{p}{\PYZcb{}}

\PY{c+cm}{/* generates a random number on [0,1)-real-interval */}
\PY{k+kt}{double} \PY{n+nf}{genrand\PYZus{}real2}\PY{p}{(}\PY{k+kt}{void}\PY{p}{)}
\PY{p}{\PYZob{}}
	\PY{k}{return} \PY{n}{genrand\PYZus{}int32}\PY{p}{(}\PY{p}{)}\PY{o}{*}\PY{p}{(}\PY{l+m+mf}{1.0}\PY{o}{/}\PY{l+m+mf}{4294967296.0}\PY{p}{)}\PY{p}{;} 
	\PY{c+cm}{/* divided by 2\PYZca{}32 */}
\PY{p}{\PYZcb{}}

\PY{c+cm}{/* generates a random number on (0,1)-real-interval */}
\PY{k+kt}{double} \PY{n+nf}{genrand\PYZus{}real3}\PY{p}{(}\PY{k+kt}{void}\PY{p}{)}
\PY{p}{\PYZob{}}
	\PY{k}{return} \PY{p}{(}\PY{p}{(}\PY{p}{(}\PY{k+kt}{double}\PY{p}{)}\PY{n}{genrand\PYZus{}int32}\PY{p}{(}\PY{p}{)}\PY{p}{)} \PY{o}{+} \PY{l+m+mf}{0.5}\PY{p}{)}\PY{o}{*}\PY{p}{(}\PY{l+m+mf}{1.0}\PY{o}{/}\PY{l+m+mf}{4294967296.0}\PY{p}{)}\PY{p}{;} 
	\PY{c+cm}{/* divided by 2\PYZca{}32 */}
\PY{p}{\PYZcb{}}

\PY{c+cm}{/* generates a random number on [0,1) with 53-bit resolution*/}
\PY{k+kt}{double} \PY{n+nf}{genrand\PYZus{}res53}\PY{p}{(}\PY{k+kt}{void}\PY{p}{)} 
\PY{p}{\PYZob{}} 
	\PY{k+kt}{unsigned} \PY{k+kt}{long} \PY{n}{a}\PY{o}{=}\PY{n}{genrand\PYZus{}int32}\PY{p}{(}\PY{p}{)}\PY{o}{>}\PY{o}{>}\PY{l+m+mi}{5}\PY{p}{,} \PY{n}{b}\PY{o}{=}\PY{n}{genrand\PYZus{}int32}\PY{p}{(}\PY{p}{)}\PY{o}{>}\PY{o}{>}\PY{l+m+mi}{6}\PY{p}{;} 
	\PY{k}{return}\PY{p}{(}\PY{n}{a}\PY{o}{*}\PY{l+m+mf}{67108864.0}\PY{o}{+}\PY{n}{b}\PY{p}{)}\PY{o}{*}\PY{p}{(}\PY{l+m+mf}{1.0}\PY{o}{/}\PY{l+m+mf}{9007199254740992.0}\PY{p}{)}\PY{p}{;} 
\PY{p}{\PYZcb{}} 
\PY{c+cm}{/* These real versions are due to Isaku Wada, 2002/01/09 added */}

\PY{k+kt}{int} \PY{n+nf}{main}\PY{p}{(}\PY{k+kt}{void}\PY{p}{)}
\PY{p}{\PYZob{}}
	\PY{k+kt}{int} \PY{n}{i}\PY{p}{;}
	\PY{k+kt}{unsigned} \PY{k+kt}{long} \PY{n}{init}\PY{p}{[}\PY{l+m+mi}{4}\PY{p}{]}\PY{o}{=}\PY{p}{\PYZob{}}\PY{l+m+mh}{0x123}\PY{p}{,} \PY{l+m+mh}{0x234}\PY{p}{,} \PY{l+m+mh}{0x345}\PY{p}{,} \PY{l+m+mh}{0x456}\PY{p}{\PYZcb{}}\PY{p}{,} \PY{n}{length}\PY{o}{=}\PY{l+m+mi}{4}\PY{p}{;}
	\PY{n}{init\PYZus{}by\PYZus{}array}\PY{p}{(}\PY{n}{init}\PY{p}{,} \PY{n}{length}\PY{p}{)}\PY{p}{;}
	\PY{n}{printf}\PY{p}{(}\PY{l+s}{"}\PY{l+s}{1000 outputs of genrand\PYZus{}int32()}\PY{l+s+se}{\PYZbs{}n}\PY{l+s}{"}\PY{p}{)}\PY{p}{;}
	\PY{k}{for} \PY{p}{(}\PY{n}{i}\PY{o}{=}\PY{l+m+mi}{0}\PY{p}{;} \PY{n}{i}\PY{o}{<}\PY{l+m+mi}{1000}\PY{p}{;} \PY{n}{i}\PY{o}{+}\PY{o}{+}\PY{p}{)} \PY{p}{\PYZob{}}
		\PY{n}{printf}\PY{p}{(}\PY{l+s}{"}\PY{l+s}{\PYZpc{}10lu }\PY{l+s}{"}\PY{p}{,} \PY{n}{genrand\PYZus{}int32}\PY{p}{(}\PY{p}{)}\PY{p}{)}\PY{p}{;}
		\PY{k}{if} \PY{p}{(}\PY{n}{i}\PY{o}{\PYZpc{}}\PY{l+m+mi}{5}\PY{o}{=}\PY{o}{=}\PY{l+m+mi}{4}\PY{p}{)} \PY{n}{printf}\PY{p}{(}\PY{l+s}{"}\PY{l+s+se}{\PYZbs{}n}\PY{l+s}{"}\PY{p}{)}\PY{p}{;}
	\PY{p}{\PYZcb{}}
	\PY{n}{printf}\PY{p}{(}\PY{l+s}{"}\PY{l+s+se}{\PYZbs{}n}\PY{l+s}{1000 outputs of genrand\PYZus{}real2()}\PY{l+s+se}{\PYZbs{}n}\PY{l+s}{"}\PY{p}{)}\PY{p}{;}
	\PY{k}{for} \PY{p}{(}\PY{n}{i}\PY{o}{=}\PY{l+m+mi}{0}\PY{p}{;} \PY{n}{i}\PY{o}{<}\PY{l+m+mi}{1000}\PY{p}{;} \PY{n}{i}\PY{o}{+}\PY{o}{+}\PY{p}{)} \PY{p}{\PYZob{}}
		\PY{n}{printf}\PY{p}{(}\PY{l+s}{"}\PY{l+s}{\PYZpc{}10.8f }\PY{l+s}{"}\PY{p}{,} \PY{n}{genrand\PYZus{}real2}\PY{p}{(}\PY{p}{)}\PY{p}{)}\PY{p}{;}
		\PY{k}{if} \PY{p}{(}\PY{n}{i}\PY{o}{\PYZpc{}}\PY{l+m+mi}{5}\PY{o}{=}\PY{o}{=}\PY{l+m+mi}{4}\PY{p}{)} \PY{n}{printf}\PY{p}{(}\PY{l+s}{"}\PY{l+s+se}{\PYZbs{}n}\PY{l+s}{"}\PY{p}{)}\PY{p}{;}
	\PY{p}{\PYZcb{}}
	\PY{k}{return} \PY{l+m+mi}{0}\PY{p}{;}
\PY{p}{\PYZcb{}}
\end{Verbatim}


\end{document}
