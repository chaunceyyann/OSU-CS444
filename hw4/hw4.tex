\documentclass[letterpaper,10pt]{article}

\usepackage{graphicx}                                        
\usepackage{amssymb}                                         
\usepackage{amsmath}                                         
\usepackage{amsthm}                                          

\usepackage{alltt}                                           
\usepackage{float}
\usepackage{color}
\usepackage{url}

\usepackage{balance}
\usepackage[TABBOTCAP, tight]{subfigure}
\usepackage{enumitem}
\usepackage{pstricks, pst-node}

\usepackage{geometry}
\geometry{textheight=8.5in, textwidth=6in}
\usepackage{booktabs}

%random comment

\newcommand{\cred}[1]{{\color{red}#1}}
\newcommand{\cblue}[1]{{\color{blue}#1}}

\newcommand{\toc}{\tableofcontents}

%\usepackage{hyperref}

%pull in the necessary preamble matter for pygments output
\input{pygments.tex}

\parindent = 0.0 in
\parskip = 0.1 in

\begin{document}

%\tableofcontents

\begin{center}
\textsc{\LARGE CS 444 Group 15}\\

% Title
{ \huge \bfseries Project 4 Group Write-up\\}

% Author
\emph{Member:}\\
Chauncey \textsc{YAN}\\
Xilun \textsc{GUO}\\
Xiaomei \textsc{WANG}\\

\end{center}

\section{What do you think the main point of this assignment is?}
The main points of this assignment is understand how the SLOB first-fit alogorithm works and implement the best-fit allocation algorithm. You must also write a program that computes the effeciency of the first-fit algorithm and best-fit algorithm and compare the fragmentation sufferred by each algorithm. This will require the addition of a system call which returns actual memory usage.
\section{How did you personally approach the problem? Design decisions, algorithm, etc.}
Personally I worked on the concurrency problem 1 and problem 2. for 1, I used pthread and semaphore to simulate three or more processes. A global variable simulate the shared resource. The threads run freely until there are three threads accessing the shared variable. another always run thread will terminate the three thread occupied the resource as the assignment descript.
For the Barber shop problem. I used the same method to present barber and customers. customer come in shop and wake up the barber. if he is working, customer has to wait. if too many customer waiting, then leave. I used the semaphore as lock to provent the deadlock situations happend, such as when consumer geting a hair cut while someone else is having a hair cut, customer come in wait with out waking up the barber. As well as run the gethaircut and cuthair functions at the same time. 

\section{How did you ensure your solution was correct? Testing details, for instance.} 
I used the print output to see all the variable value that being change while running the pthead.
\section{What did you learn?}
I learned that how to using semaphere as a lock, how to make the output colorful. how to test the program.  

\end{document}
